\chapter{Mutated checking}
\label{sec:ch4}

\info{intro}

\section{Automated checking}

The lightweight version from the previous chapter is only able to one specific
bug, the bug itself is created by manually modifying the generated system. Now,
this lightweight version needs to be automated to automatically generate a test
for every transition from a specification.

With every transition, it is possible to reach a state or stay in the current
state. To check Rebel specifications, the state to reach with a transition needs
to be defined. As mentioned before, the goal of model checking is to find a
state which is reachable with some properties which don't hold
~\cite[p.5]{stoel_storm_vinju_bosman_2016}. Thus defining only the reachable
state isn't enough, the properties of interest for a transition needs to be
specified. Each property is different per transition, so these properties should
be different for the defined state. For example with the transition closed you
want to check with the property balance with != 0 euro, and with the transition
withdraw you want to check with the property balance below zero. Basically, with
this approach we're testing the inverse of the preconditions. So we're testing
what should not be possible according to the specification.

\subsection{Evaluation points}
Evaluation points and expectations.

\section{Approach}
This is a well known approach in mutation testing. Mutation testing is
fault-based testing technique, which generates faulty programs by syntactic
changes to the original program \cite[p.1]{jia2011analysis}. The set of faulty
programs are called mutants, each mutant contains a different syntactic change.
In our case, only one mutant is generated. A test suite for a program is used to
determine whether the faulty programs are detected. A mutant is killed when it
is detected by the test suite. The mutant is in our case killed when the result
from the SMT Solver and the generated system are the same, we're using the same
approach for comparison from . Mutation testing generates a mutant based on
mutation operator, which is a transformation rule that generates a mutant from
the original program \cite[p.3-4]{jia2011analysis}. The mutation operator for
our approach is Negate Conditionals Mutator
\footnote{\url{http://pitest.org/quickstart/mutators}}, this operator belongs to
the type relational operator replacement \cite[p.688]{king1991fortran}.

\info{add image}

\subsection{Checking}
Only expressions which contain a reference to the specification fields needs to
be flipped since it is only possible in tebl to specify the reachable state with
the properties of interest (these properties are not part of the transition).
Earlier the definition of the close event transition was given in
\autoref{fig:account-close-event} which contains the following statement
\code{this.balance == EUR 0.00;}. When this statement is translated to tebl and
also with an inverse expression, it looks as follows
\code{balance != EUR 0.00;}. Thus the translated expression is just the opposite
expression of the statement defined in the close transition. Translating
expressions to inverse expressions is done for all expressions with relational
operators. The chosen mutation operator Negate Conditionals Mutator will mutate
expressions according to the replacement table in
\autoref{fig:table-replacement-conditions}.

\begin{table}[]
\centering
\begin{tabular}{cc}
\textbf{Actual expression} & \textbf{Translated expression} \\
!=                         & ==                             \\
==                         & !=                             \\
\textgreater               & \textless=                     \\
\textgreater=              & \textless                      \\
\textless                  & \textgreater=                  \\
\textless=                 & \textgreater                   \\
\end{tabular}
\caption{Flipping expressions}
\label{fig:table-replacement-conditions}
\end{table}

\subsection{Generated system}
The expressions for the transition in the generated system are also flipped.
Although, it isn't necessary to flip always the expressions. For some
transitions is an initial state expected, e.g. to execute the transition unblock
of the account specification, the account should be in the state blocked. So an
initial state needs to be constructed for some transitions. Of course, the SMT
Solver is able to construct its own initial state to reach a state.

The definition of the deposit transition is given in
\autoref{fig:account-deposit-event} and contains the following statement in the
precondition \code{amount > EUR 0.00;}. First, the initial state needs to be
constructed which is opened. Here comes the flipping of the expressions, so the
following statement from the deposit definition \code{amount > EUR 0.00;} is
flipped. Then the deposit transition needs to take place in the generated
system. Therefore, the transition parameters need to be determined based on the
flipped expression. Looking for the opposite expression with a value is for that
statement an amount less than or equal to 0 euro. For example, the following
transition parameter is generated to be used in the deposit transition
\code{"amount": "EUR -2.00"}.

\begin{sourcecode}[h!]
\begin{lstlisting}[]
event deposit(amount: Money) {
	preconditions {
		amount > EUR 0.00;
	}
	postconditions {
		new this.balance == this.balance + amount;
	}
}
\end{lstlisting}
\caption{deposit event definition from specification}
\label{fig:account-deposit-event}
\end{sourcecode}

\section{Results}
Discuss the results of the test approach.
\info{add image}

\section{Analyse}
Analyse of the results

\subsection{Codegen-Akka}
The proof of concept uses in this test run the Codegen-Akka generator.

% https://github.com/cwi-swat/ing-rebel-generators/commit/999e6b40307a79fb245ace15375c27461c92374e
\subsubsection{Bug: closing an account with balance}
\label{sec:bug-close-account}
When this automated version of checking is executed, it produces some false
positives. After investigating the tests for the transitions, the test for close
transition seems not be successful (see \autoref{fig:result-close-account}).
According to the model checker, it seems to be that the state isn't reachable
(the same tebl file is generated as in \autoref{fig:tebl-closed-account}). It
seems to be that the state is reachable in the generated system. When we take a
look at the account in the generated system, the account looks as follows in
\autoref{fig:closed-account-json}. The state of the account is in closed, which
is correct according to the specification, but the balance of the account is 52
euro. In Chapter 3 we already discussed the event definition of close in
\autoref{fig:account-close-event}, the balance should be equal to zero. From
this, we can conclude that we've discovered a bug in the generated system.

% close account

% Bugfix: == -> isEqual() instead of isNotEqual().

\begin{sourcecode}[h!]
\begin{lstlisting}[]
Test transition close
opened -> close -> closed
generated close test in |project://rebel-core/examples/simple_transaction/OpenedToClosedViaCloseTest.tebl|

Reachability transition: false
Execute transition result: true
Result successful transition test: false
\end{lstlisting}
\caption{Result run}
\label{fig:result-close-account}
\end{sourcecode}

\begin{sourcecode}[h!]
\begin{lstlisting}[]
[{
	"_id": 17592186045441,
	"_version": 2,
	"_status": "CLOSED",
	"accountNumber": {
		"iban": "NO3627716652225"
	},
	"balance": {
		"value": 52.00,
		"currency": "EUR"
	}
}]
\end{lstlisting}
\caption{account state in json}
\label{fig:closed-account-json}
\end{sourcecode}

% \begin{sourcecode}
% \begin{lstlisting}[]
% @doc {
% 	Close the account.
% }

% event close() {
% 	preconditions {
% 		this.balance == EUR 0.00;
% 	}
% }
% \end{lstlisting}
% \caption{close event definition from specification}
% \end{sourcecode}



% \begin{sourcecode}
% \begin{lstlisting}[]
% module simple_transaction.OpenedToClosedViaCloseTest

% import simple_transaction.Account

% state doCheck {
%   closed Account with balance != EUR 0.00;
% }

% check doCheck reachable in max 6 steps;
% \end{lstlisting}
% \caption{Closed account test}
% \end{sourcecode}

Now we know that we have discovered a bug, we want to know where this takes
place and if the code isn't generated correctly from the specification. The
method which handles the close transition has the following check in
\autoref{fig:java-notequal-check}. The if statement checks whether the balance
of the account is not equal to 0 euro. The condition in the if statement isn't
satisfied with the balance of 52 euro. That's why the exception
\textit{BuildCASTransactionException} isn't thrown.

\begin{sourcecode}[h!]
\begin{lstlisting}[language=Java]
if(! (isNotEqual(_entity.getBalance(), Money.of(org.joda.money.CurrencyUnit.of("EUR"), 0.00)))) {
  throw new BuildCASTransactionException("Predicate did not hold: CloseTransaction: this.balance ==
  EUR 0.00");
}
\end{lstlisting}
\caption{Code in Java}
\label{fig:java-notequal-check}
\end{sourcecode}

The question right now is, how is the above code generated.
\info{synthesis phase ref} After taking a look at the synthesization of
expression, the expressions from Rebel are not properly translated. The
synthesization for an equal expression for the type money or percentage looks as
follows in \autoref{fig:rascal-datomic-synthesize-equal}. The expression is
synthesized to the method \textit{isNotEqual} with two parameters.

\begin{sourcecode}[h!]
\begin{lstlisting}[]
private str g(e:(Expr)`<Expr lhs> == <Expr rhs>`, tmap t) = "isNotEqual(<g(lhs, t)>, <g(rhs, t)>)"
  when isType(t, lhs, (Type)`Percentage`) || isType(t, lhs, (Type)`Money`);
\end{lstlisting}
\caption{Generate equal expression in Rascal}
\label{fig:rascal-datomic-synthesize-equal}
\end{sourcecode}

So the expression isn't properly synthesized, and it should be synthesized to
\textit{isEqual} instead of \textit{isNotEqual}. With this modification, it
isn't possible anymore to close an account with some balance (this also applies
to other expressions of the type equal).

% https://github.com/cwi-swat/ing-rebel-generators/pull/6
\subsubsection{Bug: deposit with a maximum amount}
\label{sec:bug-compile-max-deposit}

The automated checking is implemented with the ability to first start the
generated system and then run the tests against it. For a new test run, the
specification has changed a little bit. It is now possible to only deposit with
a maximum amount (see \autoref{fig:java-deposit-maxamount}). After the code is
generated, the testing framework isn't able to start the system. There is a
compile error as you can see in
\autoref{fig:java-result-lessthan-compile-error}, the binary operator "$<$"
isn't applicable on the type \textit{org.joda.money.Money}. The compile error
is thrown by the source code from \autoref{fig:java-lessthan-compile-error},
which is part of the method which handles the deposit transition.

% less than, not compilable. Duplicate method for greaterThan should be lessThan

\begin{sourcecode}[h!]
\begin{lstlisting}[]
event deposit(amount: Money) {
	preconditions {
		amount < EUR 250.00;
	}
	postconditions {
		new this.balance == this.balance + amount;
	}
}
\end{lstlisting}
\caption{deposit event definition from specification}
\label{fig:java-deposit-maxamount}
\end{sourcecode}

\begin{sourcecode}[h!]
\begin{lstlisting}[]
Error:(63, 23) java: bad operand types for binary operator '<'
  first type:  org.joda.money.Money
  second type: org.joda.money.Money
\end{lstlisting}
\caption{deposit event definition from specification}
\label{fig:java-result-lessthan-compile-error}
\end{sourcecode}

\begin{sourcecode}[h!]
\begin{lstlisting}[language=Java]
if(! ((amount < Money.of(org.joda.money.CurrencyUnit.of("EUR"), 200.00)))) {
  throw new BuildCASTransactionException("Predicate did not hold: DepositTransaction:
  amount < EUR 200.00");
}
\end{lstlisting}
\caption{Code in Java}
\label{fig:java-lessthan-compile-error}
\end{sourcecode}

The synthesization of the generated is shown in
\autoref{fig:rascal-datomic-synthesize-lessthan}. Also here are the Rebel
expression not properly translated. The default expression with the binary
operator "$<$" is properly translated to an expression with two expressions with
 an the binary operator "$<$". As seen before this doesn't work with
 \textit{org.joda.money.Money}. Above this method, you can see the
 synthesization of the expression with the binary operator "$\leq$". This method
 is already defined before and this is a typical copy paste fault, the method
 should synthesize expressions with the binary operator "$<$".

% Duplicate method for greaterThan should be lessThan. This method is duplicate to the method g() on line 43. It should be
% lessThan instead of greaterThan.

\begin{sourcecode}[h!]
\begin{lstlisting}[]
private str g(e:(Expr)`<Expr lhs> \> <Expr rhs>`, tmap t) = "isGreaterThan(<g(lhs, t)>, <g(rhs, t)>)"
  when isType(t, lhs, (Type)`Percentage`) || isType(t, lhs, (Type)`Money`);
private str g(e:(Expr)`<Expr lhs> \< <Expr rhs>`, tmap t) = "(<g(lhs, t)> \< <g(rhs, t)>)";
\end{lstlisting}
\caption{Generate equal expression in Rascal}
\label{fig:rascal-datomic-synthesize-lessthan}
\end{sourcecode}

\section{Evaluation}
\label{sec:ch4-evaluation}

\subsection{Evalution points}
In this proof of concept is the account specification used to test the SUT.
This proof of concept generates automatically tests for transitions. A part of
the generated test for a transition is checking, which used to test the state to
reach with the flipped preconditions. So, in this proof of concept, we're
testing what should be not possible according to the specification. The result
of the checking from the SMT Solver varies, e.g. an opened account can be
reached by the \textit{openAccount} transition or by the transition
\textit{openAccount} and \textit{withdraw}. After the checking, a transition
is performed in the SUT. In this proof of concept, it's unknown whether the
performed transition with its parameters in the SUT is the same as the
transition computed by the SMT Solver. This causes some false positives in the
test run. Also, it is difficult to play like the SMT Solver, it is unknown which
result the SMT Solver will give and the SMT Solver is smarter/better in checking
the satisfiability of a given constraints. To conclude, the checking used in
this proof of concept tests only the states and in the SUT are transitions
tested.

With this proof of concept, we've found a bug in the generated system, which was
discussed in \autoref{sec:bug-close-account}. The other bug is out of scope
since the SUT isn't able to compile. The found bug belongs to the category
injected code since the generated code for the precondition is wrong. In this
case, the final state closed is reached where the preconditions of the close
transition don't hold. So the theory for this found bug is:
$\forall e s_{1} \to s_{2}, s_{2} \gets s_{1}, pre(e)$.

% claims from finding the bugs.

% \begin{itemize}
% \item je bent in een state gekomen via een preconditie
% \item de transitie zelf verander de pre conditie niet, (de preconditie geldt dan in de postconditie)
% \item het was de laatste

% \item $\forall e s1 \to s2, in s2 geldt pre(e) \lor (! pre(e) \land post(e))$
% \item $\forall e s1 \to s2, in s2 post(e)$
% % of?
% \item $\forall e s1 \to s2, s2 alleen bereikbaar via (1), dan pre(e)$
% \end{itemize}

\subsection{Conclusion}
Conclusion

\section{Threats to validity}
Threats to validity
