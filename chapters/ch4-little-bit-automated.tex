\chapter{Experiment 1: Invalid execution}\label{sec:ch4}

\epigraph{Discovering the unexpected is more important than confirming the
known.}{George E. P. Box}

The lightweight proof of concept discussed in \autoref{sec:ch3} is only able
to trigger one fault which is created manually. In this chapter, we discuss how
the lightweight proof of concept is automated and the test results of the
generated system.

\section{Method}

The lightweight version from the previous chapter is only able to test one specific
fault. The fault itself is created manually by modifying the \gls{sut}. Now,
this lightweight version needs to be automated to automatically generate a test
for every transition from a specification.

With every transition, it is possible to reach a state or stay in the current
state. To check \textit{Rebel} specifications, the state to reach with a transition needs
to be defined. As mentioned before, the goal of model checking is to find a
state which is reachable with some properties which do not hold
~\cite[p.~5]{stoel_storm_vinju_bosman_2016}. Thus defining only the reachable
state is not enough, the properties of interest for a transition needs to be
specified. Each property is different per transition, so these properties should
be different for the defined state. For example for the \textit{close}
transition, we want to check whether it is possible to have a closed account
where the balance is not equal to zero (as in \autoref{sec:ch3-method}),
for the transition \textit{withdraw} we want to check whether a negative balance
can be achieved with the transition.

With the lightweight version, we discussed that the model checker provides traces
only when a given state is satisfiable. When a state is not reachable, the model
checker does not provide traces. Models (traces) are not available from the Z3
solver when a given \gls{smt} problem is unsatisfiable. In this case, traces
cannot be used with opposite preconditions since they are not provided.

To conclude, with this approach we are testing the opposite of the preconditions.
Thus what is not possible according to the specification is tested.

\subsection{Evaluation criteria}\label{sec:ch4-eval-criteria}

\subsubsection{Faults}
Since we are testing with this approach the opposite of the preconditions, thus
what should be not possible according to the specification. It is expected to
find faults in the \gls{sut} where it is possible to perform the opposite of a
transition. For example, faults can be found like preconditions which are not
properly generated. An example of this is the manually created fault
(\autoref{fig:ch3-res-codegenakka-account}) for the lightweight version.

\subsubsection{Efficiency}
In this approach is checking used to check what is not possible according to the
specification. Therefore, the same tested transition should be tested in the
\gls{sut}. To test all transitions from the account specification, it may take
longer since some transactions require an initial state for which transitions
need to be performed to reach this state.

\subsubsection{Coverage}
The experiment is going to generate a test for all transitions. Therefore,
it is expected to test all the transitions of a specification. With the criteria
faults, we discussed the expectation to find faults in not properly generated
preconditions. This may lead to the inability to test transitions. For example,
when a failure (incorrect preconditions) occurs during reaching the initial
state of the \textit{withdraw} transition. This leads to the inability to test
the \textit{withdraw} transition.

\section{Approach}
The discussed testing approach is a well-known approach in mutation testing. Mutation testing is
a fault-based testing technique, which generates faulty programs by syntactic
changes to the original program.~\cite[p.~1]{jia2011analysis} The set of faulty
programs are called mutants, each mutant contains a different syntactic change.
In our case, only one mutant is generated. Mutation takes place on checking of
the specification and the execution of the transition in the \gls{sut}. A test
suite for a program is used to determine whether the faulty programs are detected.
A mutant is killed when it is detected by the test suite. The mutant is in our
case killed when the result from the \gls{smt} solver and the \gls{sut} are the
same. We are using the same approach from \autoref{sec:ch3} to compare the
results of the \gls{smt} solver and \gls{sut}.

Mutation testing generates a mutant based on
mutation operator, which is a transformation rule that generates a mutant from
the original program.~\cite[p.~3-4]{jia2011analysis} The mutation operator for
our approach is Negate Conditionals Mutator~\cite{pitmutators}, this operator
belongs to the type relational operator
replacement~\cite[p.~688]{king1991fortran}.

The testing approach is illustrated in \autoref{fig:mutated-checking}. The first
step is to start with a \textit{Rebel} specification, which is in our case the already
existing account specification. When the specifications are defined, the
specifications are being built, \textit{i.e.}, \gls{csts} are
produced of these specifications. Using these \gls{csts}, the code generator generates
the code, which is then the \gls{sut}.

The test case generator can be used to test the \gls{sut} when the \gls{sut} is generated
from the \gls{csts}. The \gls{csts} of the specifications are traversed by the test case
generator to generate a test for each transition. The test case generator
generates tebl files for transitions to use checking.

To test the \gls{sut}, the test case generator performs a similar transition
as used within checking in the \gls{sut}. Finally, the results from checking and the
performed transition in the \gls{sut} are compared.

\begin{figure}[h!]
  \centering
  \includegraphics[width=\linewidth{}]{figures/mutated-checking-diagram.pdf}
  \caption{Testing approach invalid execution}\label{fig:mutated-checking}
\end{figure}
\FloatBarrier

\subsection{Mutating checking}
Only expressions which contain a reference to the specification fields need to
be replaced since it is only possible in tebl to specify the reachable state
with the properties of interest (these properties are not part of the
transition).

Earlier the definition of the \textit{close} transition was given in
\autoref{fig:account-close-event} which contains the following statement
\code{this.balance == EUR 0.00;}. When this statement is translated to tebl
with a negated conditional, it looks as follows
\code{balance != EUR 0.00;}. Thus the replaced conditional is the opposite
condition of the statement defined in the \textit{close} transition.
Note also that the \textit{this} reference is removed, in tebl specifying
\textit{this} is not necessary since the property related to the instance.

Replacing conditionals to negated conditionals is done for all conditionals with
relational operators from \autoref{fig:table-replacement-conditions}.
The chosen mutation operator Negate Conditionals Mutator will replace
conditionals according to the replacement table in \autoref{fig:table-replacement-conditions}.

\begin{table}[h!]
\centering
\begin{tabular}{cc}
\toprule
\textbf{Actual expression} & \textbf{Translated expression} \\ \midrule
!=                         & ==                             \\
==                         & !=                             \\
\textgreater               & \textless=                     \\
\textgreater=              & \textless                      \\
\textless                  & \textgreater=                  \\
\textless=                 & \textgreater                   \\ \bottomrule
\end{tabular}
\caption{Relational conditionals replacement~\cite{pitmutators}}\label{fig:table-replacement-conditions}
\end{table}
\FloatBarrier

\subsection{Mutating transitions}

The test case generator must test the reachability in the \gls{sut} just like
the generated tebl for checking. As discussed before, the results (traces) from
checking cannot be used to check the reachability in the \gls{sut} since
traces are not available when a state is not reachable.

The conditionals for the transition in the \gls{sut} are also replaced.
Although, it is not necessary to replace always the conditionals. For some
transitions is an initial state required, \textit{e.g.}, to execute the
transition \textit{unblock} of the account specification, the account should be
in the state \textit{blocked}. So an initial state needs to be constructed for
some transitions.
Thus in constructing the initial state, it is not necessary to apply the
replacement of conditionals. Of course, with checking the \gls{smt} solver
constructs its initial state to reach a state.


For example, we now only deal with how the transition \textit{deposit} is
executed by the test case generator in the \gls{sut}, but the approach also
applies to the other transitions.
The definition of the \textit{deposit} transition is given in
\autoref{fig:account-deposit-event} and contains the following statement in the
preconditions \code{amount > EUR 0.00;}. First, the initial state needs to be
constructed which is the state opened. Therefore, the transition \textit{openAccount} is performed in the \gls{sut}.
Following is the replacement of the conditionals, the
replaced precondition from the \textit{deposit} transition looks as follows
\code{amount < EUR 0.00;}. The \textit{deposit} transition needs then to be
performed in the \gls{sut}.

To perform the \textit{deposit} transition on the \gls{sut}, the transition
parameters for this transition must be determined satisfying replaced
conditionals. The transition parameter for the \textit{deposit} transition,
amount, should be less than or equal to 0 euro. Therefore, the test case
generator picks values which satisfy the negated conditionals.
To communicate this to the \gls{sut}, this transition with its transition
parameter values must be converted to \textit{JSON}.
For example, the following transition parameter is generated in \textit{JSON} by
the test case generator to be used in the \textit{deposit} transition
\code{"amount": "EUR -2.00"}.

\begin{sourcecode}[h!]
\begin{lstlisting}[]
event deposit(amount: Money) {
	preconditions {
		amount > EUR 0.00;
	}
	postconditions {
		new this.balance == this.balance + amount;
	}
}
\end{lstlisting}
\caption{\textit{deposit} transition definition from specification}\label{fig:account-deposit-event}
\end{sourcecode}
\FloatBarrier

\section{Results}

\subsection{Codegen-Javadatomic}

For this experiment, we are testing the generator Codegen-Akka. The results
of this test run are shown in \autoref{fig:ch5-res-codegenakka-account}. As shown
in this table, the tests for four transitions are successful and the tests for
the other three transitions are failed.

\begin{table}[h!]
\centering
\begin{tabular}{cc}
\toprule
\textbf{Transition to test} & \textbf{Transition} \\ \midrule
openAccount                 & \cmark{}            \\
withdraw                    & \xmark{}            \\
deposit                     & \xmark{}            \\
interest                    & \cmark{}            \\
block                       & \cmark{}            \\
unblock                     & \cmark{}            \\
close                       & \xmark{}            \\ \bottomrule
\end{tabular}
\caption{Results: testing account specification transitions}\label{fig:ch5-res-codegenakka-account}
\end{table}
\FloatBarrier

\section{Analyse}

\subsection{Codegen-Javadatomic}

% https://github.com/cwi-swat/ing-rebel-generators/commit/999e6b40307a79fb245ace15375c27461c92374e
\subsubsection{Closing an account with balance}\label{sec:bug-close-account}
When this automated version of checking is executed, it produces some false
positives. After investigating the tests for the transitions, the test for the \textit{close}
transition seems not be successful (see \autoref{fig:result-close-account}).
On line 6 is shown that the model checker states that the state is not reachable (the same tebl file is
generated as in \autoref{fig:tebl-closed-account}). On the next line, it seems to be that the
state is reachable in the \gls{sut}. So the test for \textit{close} transition
is not successful.

\begin{sourcecode}[h!]
\begin{lstlisting}[]
Test transition close
opened -> close -> closed
generated close test in |project://rebel-core/examples/simple_transaction/
  OpenedToClosedViaCloseTest.tebl|

Reachability transition: false
Execute transition result: true
Result successful transition test: false
\end{lstlisting}
\caption{Results: test run for the \textit{close} transition}\label{fig:result-close-account}
\end{sourcecode}
\FloatBarrier

When we take a look at the account in the \gls{sut}, it looks as
follows in \autoref{fig:closed-account-json}. The state of the account is in
\textit{closed}, which is correct according to the specification, but the
balance of the account is 52 euro. In \autoref{fig:account-close-event} we
already discussed the transition definition of the \textit{close} transition,
which is that the balance should be equal to zero. From this, we can conclude
that we have discovered a fault in the \gls{sut}.

\begin{sourcecode}[h!]
\begin{lstlisting}[]
[{
	"_id": 17592186045441,
	"_version": 2,
	"_status": "CLOSED",
	"accountNumber": {
		"iban": "NO3627716652225"
	},
	"balance": {
		"value": 52.00,
		"currency": "EUR"
	}
}]
\end{lstlisting}
\caption{Account state after \textit{close} transition}\label{fig:closed-account-json}
\end{sourcecode}
\FloatBarrier

Now we know that we have discovered a fault, we want to know why this behaviour
occurs and whether it is due to the generated code from the specification. The
method which handles the \textit{close} transition has the following check in
\autoref{fig:java-notequal-check}. The if statement checks whether the balance
of the account is not equal to 0 euro. The condition in the if statement is not
satisfied with the balance of 52 euro. That is why the exception
\textit{BuildCASTransactionException} is not thrown.

\begin{sourcecode}[h!]
\begin{lstlisting}[language=Java]
if(! (isNotEqual(_entity.getBalance(), Money.of(org.joda.money.CurrencyUnit.of("EUR"), 0.00)))) {
  throw new BuildCASTransactionException("Predicate did not hold: CloseTransaction: this.balance ==
  EUR 0.00");
}
\end{lstlisting}
\caption{Generated precondition for the \textit{close} transition}\label{fig:java-notequal-check}
\end{sourcecode}
\FloatBarrier

The question right now is, how is the above code generated. After taking a look at the synthesization of
expression, the expressions from \textit{Rebel} are not properly synthesized. The
synthesization for an equal expression for the type \textit{Money} or \textit{Percentage} looks as
follows in \autoref{fig:rascal-datomic-synthesize-equal}. The expression is
synthesized to the method \textit{isNotEqual} with two parameters.

\begin{sourcecode}[h!]
\begin{lstlisting}[]
private str g(e:(Expr)`<Expr lhs> == <Expr rhs>`, tmap t) = "isNotEqual(<g(lhs, t)>, <g(rhs, t)>)"
  when isType(t, lhs, (Type)`Percentage`) || isType(t, lhs, (Type)`Money`);
\end{lstlisting}
\caption{Equals expression generator}\label{fig:rascal-datomic-synthesize-equal}
\end{sourcecode}
\FloatBarrier

So the expression is not properly synthesized, and it should be synthesized to
\textit{isEqual} instead of \textit{isNotEqual}. With this modification, it
is not possible anymore to close an account with some balance. This also applies
to other statements which use the equal operator.

% https://github.com/cwi-swat/ing-rebel-generators/pull/6
\subsubsection{Deposit with a maximum amount}\label{sec:bug-compile-max-deposit}

The automated checking is implemented with the ability to first start the
\gls{sut} and then run the tests against it. For a new test run, the
specification has changed a little bit. It is now possible to only deposit with
a maximum amount (see \autoref{fig:java-deposit-maxamount}). After the code is
generated, the testing framework is not able to start the system. There is a
compile error as you can see in
\autoref{fig:java-result-lessthan-compile-error}, the binary operator "$<$"
is not applicable on the type \textit{org.joda.money.Money}. The compile error
is thrown by the source code from \autoref{fig:java-lessthan-compile-error},
which is part of the method which handles the \textit{deposit} transition.

% less than, not compilable. Duplicate method for greaterThan should be lessThan

\begin{sourcecode}[h!]
\begin{lstlisting}[]
event deposit(amount: Money) {
	preconditions {
		amount < EUR 250.00;
	}
	postconditions {
		new this.balance == this.balance + amount;
	}
}
\end{lstlisting}
\caption{\textit{deposit} transition definition from specification}\label{fig:java-deposit-maxamount}
\end{sourcecode}
\FloatBarrier

\begin{sourcecode}[h!]
\begin{lstlisting}[]
Error:(63, 23) java: bad operand types for binary operator '<'
  first type:  org.joda.money.Money
  second type: org.joda.money.Money
\end{lstlisting}
\caption{Compile error in generated system}\label{fig:java-result-lessthan-compile-error}
\end{sourcecode}
\FloatBarrier

\begin{sourcecode}[h!]
\begin{lstlisting}[language=Java]
if(! ((amount < Money.of(org.joda.money.CurrencyUnit.of("EUR"), 200.00)))) {
  throw new BuildCASTransactionException("Predicate did not hold: DepositTransaction:
  amount < EUR 250.00");
}
\end{lstlisting}
\caption{Generated precondition for the \textit{deposit} transition}\label{fig:java-lessthan-compile-error}
\end{sourcecode}
\FloatBarrier

The functions for the synthesization, which generates a part of
\autoref{fig:java-lessthan-compile-error}, are shown in
\autoref{fig:rascal-datomic-synthesize-lessthan}. Also here are the \textit{Rebel}
expression not properly synthesized. The default expression with the binary
operator "$<$" is properly synthesized to an expression with three expressions,
the left-hand and right-hand side expression and the binary operator "$<$".
As discussed before, the binary operator "$<$" doesn't work with
\textit{org.joda.money.Money}. Thus the default method to synthesize expressions
with the binary operator "$<$" cannot be used for the type
\textit{org.joda.money.Money}.

On line number 1 of \autoref{fig:rascal-datomic-synthesize-lessthan} is the
synthesization method of the expression with the binary operator "$>$" shown.
This method is already defined before in the corresponding file. To conclude,
this method should synthesize expressions with the binary operator "$<$".

\begin{sourcecode}[h!]
\begin{lstlisting}[]
private str g(e:(Expr)`<Expr lhs> \> <Expr rhs>`, tmap t) = "isGreaterThan(<g(lhs, t)>, <g(rhs, t)>)"
  when isType(t, lhs, (Type)`Percentage`) || isType(t, lhs, (Type)`Money`);
private str g(e:(Expr)`<Expr lhs> \< <Expr rhs>`, tmap t) = "(<g(lhs, t)> \< <g(rhs, t)>)";
\end{lstlisting}
\caption{GreaterThan and LessThan expression generator}\label{fig:rascal-datomic-synthesize-lessthan}
\end{sourcecode}
\FloatBarrier

\section{Evaluation}\label{sec:ch4-evaluation}

\subsection{Faults}
In \autoref{sec:ch4-eval-criteria} we discussed the expectations of the criteria
faults. We expected to find faults in the \gls{sut} where it is possible to perform
the opposite of a transition. Thus it is expected to find faults where the
preconditions are not properly generated.

With this experiment, we have found a fault in the \gls{sut}, which was
discussed in \autoref{sec:bug-close-account}. The other fault is out of scope
since the \gls{sut} is not able to compile. With the fault from
\autoref{sec:bug-close-account}, it is possible that the final state closed is
reached where the preconditions of the \textit{close} transition do not hold.
So as expected, we did find a fault in performing the opposite of a transition
where the preconditions were not properly generated.

In this experiment, traces are not used because they cannot be provided by the
\gls{SMT} solver when a state is not reachable. The expectation is that with
testing the opposite preconditions that the traces are not provided.
The reasons for this is that with the opposite preconditions that the state is
not reachable, and when the state to reach is not reachable traces cannot be
provided by the model checker.
Remarkable is that with testing some transition, the traces are provided
because the state to reach with checking are reachable. For example, the
\textit{block} transition has no precondition which means that the state to
reach is reachable with checking.

\subsection{Efficiency}
For the criterion efficiency, it is expected to check what is not possible
according to the specification, i.e. testing the same transition in checking as
well as in the \gls{sut}.

A part of the generated test for a transition is checking, which is used to test
the state to reach with the replaced preconditions. So, in this experiment, we
are testing what should be not possible according to the
specification. The expectation is that the same transitions with checking should
be performed on the \gls{sut}. However, the result of the checking from the
\gls{smt} solver varies, \textit{e.g.}, an opened account can be reached by the
\textit{openAccount} transition or by the transition \textit{openAccount} and
\textit{withdraw}. This can be limited by taking a lower configuration timeout
in checking. Mainly it remains that with checking it is not possible to focus on
a specific transition. Thus the test framework is not able to perform the same
transitions on the \gls{sut} as the transitions from checking.

With testing all transitions from the account specification, it is possible that
testing may take longer. As expected, this is the case since due to the initial
state transitions are more executed and tested. To conclude, the testing process
may take longer to test all the transitions.

\subsection{Coverage}
It is expected for this criterion to test all the transitions of the
specification since the experiment generated tests for all transitions.

In the experiment, after the checking, a transition
is performed in the \gls{sut}. In this experiment, it is unknown whether the
performed transition with its parameters in the \gls{sut} is the same as the
transition computed by the \gls{smt} solver. This causes some false positives in the
test run. Also, it is difficult to play like the \gls{smt} solver; it is unknown which
result the \gls{smt} solver will give. The \gls{smt} solver is also smarter/better in checking
the satisfiability of a given constraint.

Failure occurring along the way in constructing the initial state of a transition
may lead to the inability to test transitions. Unfortunately, there does not
seem to be any faults in here.

\section{Conclusion}
This experiment uses the account specification to test the \gls{sut}.
This experiment generates automatically tests for transitions.

A part of
the generated test for a transition is checking, which is used to test the state to
reach with the replaced preconditions. So, in this experiment, we are
testing what should be not possible according to the specification. The result
of the checking from the \gls{smt} solver varies, \textit{e.g.}, an opened account can be
reached by the \textit{openAccount} transition or by the transition
\textit{deposit}, \textit{withdraw} and \textit{interest}.

After the checking, a transition
is performed in the \gls{sut}. In this experiment, it is unknown whether the
performed transition with its parameters in the \gls{sut} is the same as the
transition computed by the \gls{smt} solver. This causes some false positives in
the test run. Also, it is difficult to play like the \gls{smt} solver. It is
unknown which result the \gls{smt} solver will give, mainly because it remains
that with checking it is not possible to focus on a specific transition. The
\gls{smt} solver is also smarter/better in checking the satisfiability of a
given constraint.

To conclude, the checking used in this experiment tests only the states, regardless of which transitions are
being performed, and testing the \gls{sut} focuses more on testing transitions.

With this experiment, we have found a fault in the \gls{sut}, which was
discussed in \autoref{sec:bug-close-account}. The other fault is out of scope
since the \gls{sut} is not able to compile. The found fault belongs to the category
injected code since the generated code for the precondition is wrong. In this
case, the final state closed is reached where the preconditions of the \textit{close}
transition do not hold.

\section{Threats to validity}

\subsection*{Limited specifications}
In the conducted experiment is the account specification account used
to test the \gls{sut}. With this experiment and specification,
we did find a fault in the code generators.

The used account specification in this experiment is quite simple. With the use
of more interacting specifications, the chance is bigger to find faults in the code
generators since the specifications are interacting with each other.

\subsection*{Invalid execution trace}
The conducted experiment test only what should be not possible according to the
specification. It is also important to test whether the \gls{sut} is conform to the
specification, \textit{i.e.}, testing the valid execution trace.
Testing valid execution can use traces as these states are reachable.
