\chapter{Conclusion}\label{sec:ch7}

In this work, we have shown two proof of concepts to test generated systems from
\textit{Rebel} specifications. With these proof of concepts, it can be tested
whether the generated systems are generated properly based on \textit{Rebel}
specifications. The result of this is that we regained the benefits from
\textit{Rebel} domain, and again able to test and reason about \textit{Rebel}
specifications and generated system.

\begin{quote}
  \textbf{How to validate the generated code from a Rebel specification?}
\end{quote}

In an earlier study~\cite[p.3]{stoel2015case}, the author proposed a possible
solution, which is to use the \gls{smt} solver to test generated systems.
In both proofs of concepts, the \gls{smt} solver holds the key in testing the
generated systems.

The generated systems are tested in two ways, invalid execution and valid execution.
The first experiment tests invalid execution in the generated systems, \textit{i.e.},
testing what should be not possible according to the specification.
Therefore, the test framework uses checking to check the satisfiability for a
transition from the specification, and then test this in the \gls{sut}.
In this experiment is a mutation operator, which is used in mutation testing,
applied to test invalid execution. With this experiment, we did find two faults in
the \gls{sut}; only one fault is within the research scope.

The second experiment tests valid execution in the generated systems,
\textit{i.e.}, testing what should be possible according to the specification.
This experiment uses two existing testing techniques within \textit{Rebel} to
generate tests for transitions, namely checking and simulation.
The traces from this are used to test \gls{sut} in this experiment.
In this experiment is a model-based testing approach taken to check whether the
\gls{sut} accept the execution from traces whether the \gls{sut} behaves as the
specification.
Even the transition parameters data values are generated by the SMT solver which
satisfies the constraints of the transition. With this experiment, we did find
three faults in the \gls{sut}; one fault is identified by a slightly different test
approach to find distribution faults.

To sum up, with the experiments a total of five faults have been found in the
generated system that is generated by the code generators.
These faults can be categorised in the following categories: templating,
compilation and distribution.

\section{Future work}

\subsection{mutation model-based testing}
